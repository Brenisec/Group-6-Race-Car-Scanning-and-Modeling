\documentclass[draftclsnofoot,onecolumn,letterpaper,10pt,compsoc]{IEEEtran}
\usepackage[utf8]{inputenc}
\usepackage[T1]{fontenc}
% \usepackage[backend=bibtex]{biblatex}
\usepackage[english]{babel}
\usepackage[margin=0.75in]{geometry}
\usepackage[pdf]{pstricks}
\usepackage[utf8]{inputenc}
\usepackage{alltt}                                           
\usepackage{amsmath}                                         
\usepackage{amssymb}                                         
\usepackage{amsthm}                                          
\usepackage{blindtext}
\usepackage{calc}
\usepackage{caption}
\usepackage{color}
\usepackage{csquotes}
\usepackage{enumitem}
\usepackage{float}
\usepackage{geometry}
\usepackage[xindy]{glossaries}
\makeglossaries
\usepackage{graphicx}
\usepackage{hyperref}
\usepackage{listings}
\usepackage{longtable}
\usepackage{pst-gantt}
\usepackage{pst-uml}
\usepackage{subcaption}
\usepackage{titlesec}
\usepackage{titling}
\usepackage{url}
\usepackage{pdfpages}
\usepackage{array}

\newcommand{\subtitle}[1]{%
  \posttitle{%
    \par\end{center}
    \begin{center}\large#1\end{center}
    \vskip0.5em}%
}
\begin{document}

\title{\huge Problem Statement}
\subtitle{CS 461 - CS Senior Capstone \\ Race Car Scanning and Modeling}
\author{Donghao Lin\\ Joshua Diedrich \\Christopher Breniser \\ }
\date{Fall 2018}
\maketitle
\vspace{2cm}
\begin{abstract}

\noindent Indie Lights racing is a dangerous sport and as such, needs safety standards to mitigate that danger whenever possible. We will be working with REHV to develop a process of scanning vehicles prior to participating in a race to ensure that they meet exterior body compliance measurements. The traditional method of measurements was by manually checking each dimension and recording it by hand. The proposed solution reduce the number of hand measurements that must occur to speed up the inspection process. We'll be designing a software to take physical measurements about each vehicle using lasers to check specific parts of the body of a car. That data will then be stored to a local device that can then organize the data in an exportable format that can be easily transferred where needed.  
\end{abstract}
\clearpage
\section*{Problem Definition}
This project effects REHV  and the software they provide, TechWorks. In the field of racing, strict standards must be met to address safety. To pass compliance and be confirmed ready for a race, a vehicle must pass several inspections. These must be performed on every participating car before they are approved. The issue is that car inspections are time consuming and costly. Each measurement must be done by hand. As well, the data once gathered is often not in digital form, making it hard to move around unless hand inputted into a database. This is not cost effective and leads to more time being put into data gathering and input than is necessary. 

\noindent This project will address the fact that time is very precious before a race. There are many different tasks to be done and the setup for and time taken to do each major task is lengthy. We aim to alleviate the pressure of rushing inspections by speeding up the laborious job that is measuring the dimensions of a car body and documenting that information as well.  
\section*{Proposed Solution}
    
To go about solving this problem, we will start by setting up approximately multiple cameras around a vehicle to take photos from different angles. These cameras will be spaced a predetermined distance from one another, and from the vehicle itself. This will allow us to analyze those photos to create a measurable rendered image of sorts. From this render we will be able to take necessary measurements based on the cameras locations and distance from the vehicle, giving accurate measurements for that vehicle. The images will be taken from just one side and the top of the vehicle with the assumption that the vehicle is symmetrical. These measurements will then be stored in a local file system to which they can be approved and accessed by other TechWorks software needing that data. This camera array will be fitted with coverings over necessary components to alleviate issues brought up by rain, sun and other elements one would expect to see at an outdoor racetrack.
    
\noindent The collected data will be send to a local server on site and that information will be stored in the TechWorks software where it can be accessed by techs on site via tablets hooked to the local server over wireless. We are simply integrating the measuring process into the already existing server and communications software distributed to all relevant devices. This will allow a quick method to gather relevant measurement data and disperse it quickly to members for approval and compliance checking.

\section*{Performance Metrics}
Once finished, there should be a cost effective save in time to vehicle exterior body measuring and the data collected should be easy to manage and work with. This means having it be displayed to the user cleanly and easily exported to any user using TechWock software that needs to access it.
    
\noindent The measuring of the vehicle should be mostly automated with some user assistance in setting up. This must be deployable before a race and must be viable as a time saver as a replacement for measuring by hand. Deployment should be simple and resettable for multiple cars. The deployment would ideally not take up too much space and would be integrated into the setup of the pre-race inspection team.
    
\noindent These inspections should be able to take place on the field near the tarmac. They should be able to withstand expected weather conditions such as rain, wind and long periods of sun during use. Sun should not play a factor in our cameras effectiveness when scanning.  


\noindent The information gathered should be linked to the vehicle via the TechWork’s app and that data should follow the car through its relevant lifespan in that race. The data collected will be automatically stored and sent over a wireless connection to a local sever so the measurements can be accessed by any authorized to check them.
\clearpage

% \begin{thebibliography}{9}
% \bibitem{latexcompanion} 
% Michel Goossens, Frank Mittelbach, and Alexander Samarin. 
% \textit{The \LaTeX\ Companion}. 
% Addison-Wesley, Reading, Massachusetts, 1993.
 
% \bibitem{einstein} 
% Albert Einstein. 
% \textit{Zur Elektrodynamik bewegter K{\"o}rper}. (German) 
% [\textit{On the electrodynamics of moving bodies}]. 
% Annalen der Physik, 322(10):891–921, 1905.
 
% \bibitem{knuthwebsite} 
% Knuth: Computers and Typesetting,
% \\\texttt{http://www-cs-faculty.stanford.edu/\~{}uno/abcde.html}
% \end{thebibliography}

\end{document}
